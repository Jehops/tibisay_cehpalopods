% Created 2019-08-27 Tue 14:30
% Intended LaTeX compiler: pdflatex
\documentclass[11pt]{article}
\usepackage[utf8]{inputenc}
\usepackage[T1]{fontenc}
\usepackage{graphicx}
\usepackage{grffile}
\usepackage{longtable}
\usepackage{wrapfig}
\usepackage{rotating}
\usepackage[normalem]{ulem}
\usepackage{amsmath}
\usepackage{textcomp}
\usepackage{amssymb}
\usepackage{capt-of}
\usepackage{hyperref}
\author{Joseph Mingrone}
\date{\today}
\title{}
\hypersetup{
 pdfauthor={Joseph Mingrone},
 pdftitle={},
 pdfkeywords={},
 pdfsubject={},
 pdfcreator={Emacs 27.0.50 (Org mode 9.2.5)}, 
 pdflang={English}}
\begin{document}

\tableofcontents

\section{Outline of Cephalopod Study}
\label{sec:orgd31c6fc}
\subsection{Branch-Site Simulations}
\label{sec:org9701120}
\subsubsection{Simulations under \(H_0\)}
\label{sec:orgf649161}
\begin{itemize}
\item All simulations except 1. use an 8-taxon tree
\item Simulations 1-4 used 500-codon sequences and simulations 5-11 used
5000-codon sequences
\item All Simulations were with 1000 replicates
\end{itemize}

\href{https://ftfl.ca/misc/bsa/bsa\_sim1\_4-taxon-tree.pdf}{1.} One foreground branch at a tip of symmetric 4-taxon tree; total tree
length 3; \(p_0=0.75\), \(p_1=0.21\), \(\omega_0=0.4\)
\begin{itemize}
\item Conservative behaviour unless we condition on \(\omega_2>1\), in which case
we match a \(\chi^2_1\)
\item Distribution of MLEs look good except some inflated \(\omega_2\) estimates
\end{itemize}

\href{https://ftfl.ca/misc/bsa/bsa\_sim2\_8-taxon-tree.pdf}{2.} One foreground branch at a tip; total tree length 3; \(p_0=0.7\),
\(p_1=0.2\), \(\omega_0=0.3\)
\begin{itemize}
\item Conservative behaviour unless we condition on \(\omega_2>1\), in which case
we match a \(\chi^2_1\)
\item Distribution of MLEs look good except some inflated \(\omega_2\) estimates
\end{itemize}

\href{https://ftfl.ca/misc/bsa/bsa\_sim3\_8-taxon-tree.pdf}{3.} One foreground branch at a tip; total tree length 3; \(p_0=0.75\),
\(p_1=0.25\), \(\omega_0=0.3\)
\begin{itemize}
\item Clear conservative behaviour unless we condition on \(\omega_2>1\), in which
case we match a \(\chi^2_1\)
\item Distribution of MLEs look good except some inflated \(\omega_2\) estimates
\end{itemize}

\href{https://ftfl.ca/misc/bsa/bsa\_sim4\_8-taxon-tree.pdf}{4.} One foreground branch at a tip; total tree length 3; \(p_0=0.25\),
\(p_1=0.75\), \(\omega_0=0.3\)
\begin{itemize}
\item Slight conservative behaviour unless we condition on \(\omega_2>1\), in
which case we nearly match a \(\chi^2_1\) (slightly anti-conservative
relative to \(\chi^2_1\) distribution)
\item Distribution of MLEs look good except some inflated \(\omega_2\) estimates
\end{itemize}

\href{https://ftfl.ca/misc/bsa/bsa\_sim5\_8-taxon-tree.pdf}{5.} One foreground branch at a tip; foreground branch is 1/10 the length of
the other branches; total tree length 6; \(p_0=0.5\), \(p_1=0.5\), \(\omega_0=0\)
\begin{itemize}
\item Very conservative behaviour, even when we condition on \(\omega_2>1\) (386
of the \(\omega_2\) estimates).
\item There are left tails on the \(p_0\) and \(p_1\) MLE distributions and right
tails on the p₂ₐ and p₂b and \(\omega_2\) MLEs.
\end{itemize}

\href{https://ftfl.ca/misc/bsa/bsa\_sim6\_8-taxon-tree.pdf}{6.} One foreground branch at a tip; foreground branch 10x the length of the
other branches; total tree length 6; \(p_0=0.5\), \(p_1=0.5\), \(\omega_0=0\)
\begin{itemize}
\item Still very conservative behaviour, even when we condition on \(\omega_2>1\)
(924 of the \(\omega_2\) estimates).
\item Distribution of MLEs look good except some inflated \(\omega_2\) estimates
\end{itemize}

\href{https://ftfl.ca/misc/bsa/bsa\_sim7\_8-taxon-tree.pdf}{7.} Half of tree in foreground; total tree length 6; \(p_0=0.5\), \(p_1=0.5\),
\(\omega_0=0\)
\begin{itemize}
\item Anti-conservative behaviour; conditioning on \(\omega_2>1\) (674) shows
slight anti-conservative behaviour relative to \(\chi^2_1\)
\item MLEs look good (boundary issues); some inflated \(\omega_2\) estimates
\end{itemize}

\href{https://ftfl.ca/misc/bsa/bsa\_sim8\_8-taxon-tree.pdf}{8.} Half of tree in foreground; total tree length 6; \(p_0=0.475\),
\(p_1=0.475\), \(\omega_0=0\)
\begin{itemize}
\item Anti-conservative behaviour; conditioning on \(\omega_2>1\) (553) shows
anti-conservative behaviour relative to \(\chi^2_1\)
\item MLEs look good (boundary issues only with ωₒ); no inflated \(\omega_2\)
estimates
\end{itemize}

\href{https://ftfl.ca/misc/bsa/bsa\_sim9\_8-taxon-tree.pdf}{9.} Half of tree in foreground; total tree length 6; \(p_0=0.375\),
\(p_1=0.375\), \(\omega_0=0\)
\begin{itemize}
\item Anti-conservative behaviour; conditioning on \(\omega_2>1\) (531) shows
anti-conservative behaviour relative to \(\chi^2_1\)
\item MLEs look good (boundary issues only with ωₒ); no inflated \(\omega_2\)
estimates
\end{itemize}

\href{https://ftfl.ca/misc/bsa/zbsa\_sim10\_8-taxon-tree.pdf}{10.} One internal branch in foreground 1/10th the length; total tree length
6; \(p_0=0.5\), \(p_1=0.5\), \(\omega_0=0\)
\begin{itemize}
\item Very conservative behaviour; conditioning on \(\omega_2>1\) (571) shows very
conservative behaviour as well.
\item MLEs are wide (\(p_0/p_1\) long left tails, \(p_{2a}/p_{2b}\) long right
tails); \(\omega_2\) has inflated estimates
\end{itemize}

\href{https://ftfl.ca/misc/bsa/zbsa\_sim11\_8-taxon-tree.pdf}{11.} One internal branch in foreground 10x the length; total tree length 6;
\(p_0=0.5\), \(p_1=0.5\), \(\omega_0=0\)
\begin{itemize}
\item Still conservative behaviour, but less so with the longer branch;
conditioning on \(\omega_2>1\) (421) shows only very mild conservative
behaviour relative to \(\chi^2_1\) mixture.
\item Weight MLEs are bimodal; \(\omega_2\) has inflated estimates
\end{itemize}

\subsubsection{Notes}
\label{sec:orgb860af1}
\begin{itemize}
\item Single branch are not sufficiently influential to cause anti-conservative
behaviour
\begin{itemize}
\item We see anti-conservative behaviour with information sparseness
\end{itemize}
\item With more of the tree in the foreground, we see similar behaviour to M2a
\item If we restrict to simulations with \(\omega_2>1\), we often match \(\chi^2_1\)
\item With one foreground branch, the models are so conservative, a
re-calibration may be called for
\item The LR statistics distributions looks very different depending on how the
foreground is chosen.  This is a problem.
\item Penalization could still be relevant when the foreground takes up more of
the tree.
\item When \(p_0+p_1<1\), the theory of Self and Liang applies and the LRS
distributions should match a χ² mixture.  Perhaps running simulation 9
with penalization to support this.
\item When \(p_0+p_1=1\), the theory of Self and Liang does not apply (why) and we
expect conservative behaviour.
\item \(p_0+p_1=1\) is the better \(H_0\) due to adaptive evolution (from a MutSel
perspective).  This can occur when going from \(\omega<1\) to\ldots{}??.  That's
exactly what \(p_{2a}\) corresponds to.
\item Inflated \(\omega_2\) is still a problem.
\item Corrections for multiple tests assume that the probability of a type I
error is the same for each test, but that's not the case here.
\end{itemize}

\subsection{Data}
\label{sec:org85573b2}
\begin{itemize}
\item \href{Dal\#CANiYCztS-ihazv3UnjOxOANHK35NT4Lf0OKL2\_gkaq5mdMHBiA@mail.gmail.com}{Email from Tibisay Escalona about Cephalopad Data}
\item \href{file:///home/jrm/scm/tibisay\_cephalopods/data/Data\_28May\_2019\_TE/}{Cephalopod Data and Analyses}
\end{itemize}

There are 35 taxa with 3738 codons (3539 without ambiguous sites).  Each of
the 10 branch-site tests is on a single branch.

\subsection{Branch-Site Analyses}
\label{sec:org993a25f}

\href{https://ftfl.ca/misc/bsa/cephalopod\_brs\_summary.pdf}{Summary of Branch-Site Analyses of Celphalopod Data}

The LR tests for \#2, 4, 5, 6, and 8 (Groenlandibelus, Loliginidae,
Oegop\textsubscript{Bathy}, and Oegopsida, Sepiida) were significant after correcting for
multiple tests.

\subsection{SBA Analyses}
\label{sec:org55045a0}

\href{https://ftfl.ca/misc/bsa/cephalopod\_sba\_summary.pdf}{Summary of SBA Analyses of Cephalopod Data}

The SBA analyses were run on the 5 branches that were significant under the
branch-site tests.  The \(\omega_2\) estimates tended towards infinity (they
were all 999 for Groenlandibelus and Sepiida).  The other MLE estimates
looked stable, except for some instability with Oegop\textsubscript{Bathy} and Oegopsida
branches.


\subsection{Proposed Simulations}
\label{sec:org41104f2}

Simulate 35 taxa 3500 codons using tree topology and foreground branches from
Cephalopod data both under the null and with various ω>1.  We could also run
these under SBA to discover MLE distributions.

\subsection{Planned Tests to assess Biological Conclusions}
\label{sec:org958b8ee}
\href{nnml:Dal\#93BCA37A-0B0B-422E-8C39-09ED9A419A2F@dal.ca}{Email from Joe with two recent references for some of these tests} \href{file:///home/jrm/files/edu/papers/Improved\_inference\_of\_site-specific\_positive\_selection\_under\_a\_generalized\_parametric\_codon\_model\_when\_there\_are\_multinucleotide\_mutations\_and\_multiple\_nonsynonymous\_rates\_-\_Dunn\_et\_al\_-\_2019\_-\_BMC\_Evolutionary\_Biology\_.pdf}{Joe and
Kathy's paper about multinucleotide mutations and multiple nonsynonymous
rates}
\begin{itemize}
\item test for double/triple (DT) nucleotide changes
\item test for recombination
\item test for variation in dS
\item rerun the tests for Muse and Gaut (option 5 in codeml), which models
transition probabilities in the Q matrix as a proportion to the target
nucleotide
\end{itemize}
\end{document}